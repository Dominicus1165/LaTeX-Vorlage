\chapter{Kapiteltitel}
Etwas Text.
\section{Beispiel Unterkapitel}
Etwas Text.
\subsection{Beispiel Unterunterkapitel}
Etwas Text.
\subsubsection{Beispiel Unterunterunterkapitel}
Etwas Text.
\paragraph{Beispiel Unterunterunterunterkapitel}         % unnummeriert
Etwas Text.
\subparagraph{Beispiel Unterunterunterunterunterkapitel} % unnummeriert
Etwas Text.
\section*{Unterkapitel nicht in Inhaltsverzeichnis}      % nicht im Inhaltsverzeichnis
Etwas Text.

Mit einem \texttt{*} kann man Dinge nicht im Inhaltsverzeichnis aufführen.\\
Das ist für das \emph{Abstract} relevant, welches mit \texttt{\textbackslash chapter*}\(\{\texttt{Abstract}\}\) eingeleitet wird.

Zitat\cite{zitat}\\
Zitat\cite[10-20]{zitat}

\newpage

Abkürzungen kommen ins \texttt{glossary.tex}.\\
Dies geht mit \texttt{\acrfull{Bsp}}\\
\acrfull{Bsp} oder \acrfullpl{Bsp} kommen zuerst,\\
dann \acrshort{Bsp} oder \acrlongpl{Bsp}.

\begin{listing}[H]
    \inputminted{python}{code/beispielcode.py}
    \caption{Hier etwas Beispielcode}
    \label{lst:beispielcode.py}
\end{listing}
Außerdem noch eine Formel \ref{fkt:Beispielformel}.

\begin{equation}
    \sin \alpha = \frac{a}{c}
    \label{fkt:Beispielformel}
\end{equation}

Und ein Bild \ref{img:Beispielbild} darf auch nicht fehlen.

\begin{figure}[H]
    \centering
    \includegraphics[width=0.7\textwidth]{images/Beispielbild.png}
    \caption{Beispielbild}
    \label{img:Beispielbild}
\end{figure}